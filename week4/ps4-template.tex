%
% 6.006 problem set 4 solutions template
%
\documentclass[12pt,twoside]{article}

\input{macros-sp20}
\newcommand{\theproblemsetnum}{4}

\title{6.006 Problem Set \theproblemsetnum}

\begin{document}

\handout{Problem Set \theproblemsetnum}

\setlength{\parindent}{0pt}
\medskip\hrulefill\medskip

{\bf Name:} Your Name

\medskip

{\bf Collaborators:} Name1, Name2

\medskip\hrulefill

%%%%%%%%%%%%%%%%%%%%%%%%%%%%%%%%%%%%%%%%%%%%%%%%%%%%%
% See below for common and useful latex constructs. %
%%%%%%%%%%%%%%%%%%%%%%%%%%%%%%%%%%%%%%%%%%%%%%%%%%%%%

% Some useful commands:
%$f(x) = \Theta(x)$
%$T(x, y) \leq \log(x) + 2^y + \binom{2n}{n}$
% {\tt code\_function}


% You can create unnumbered lists as follows:
%\begin{itemize}
%    \item First item in a list
%        \begin{itemize}
%            \item First item in a list
%                \begin{itemize}
%                    \item First item in a list
%                    \item Second item in a list
%                \end{itemize}
%            \item Second item in a list
%        \end{itemize}
%    \item Second item in a list
%\end{itemize}

% You can create numbered lists as follows:
%\begin{enumerate}
%    \item First item in a list
%    \item Second item in a list
%    \item Third item in a list
%\end{enumerate}

% You can write aligned equations as follows:
%\begin{align}
%    \begin{split}
%        (x+y)^3 &= (x+y)^2(x+y) \\
%                &= (x^2+2xy+y^2)(x+y) \\
%                &= (x^3+2x^2y+xy^2) + (x^2y+2xy^2+y^3) \\
%                &= x^3+3x^2y+3xy^2+y^3
%    \end{split}
%\end{align}

% You can create grids/matrices as follows:
%\begin{align}
%    A =
%    \begin{bmatrix}
%        A_{11} & A_{21} \\
%        A_{21} & A_{22}
%    \end{bmatrix}
%\end{align}

% You can include images and PDFs as follows:
% \includegraphics[width=0.5\textwidth]{img.jpg}

\begin{problems}

\problem  % Problem 1

\begin{problemparts}
\problempart % Problem 1a
node 16 and 37 is not height balanced with skew 2 and -2 respectively.
\problempart % Problem 1b

{\tt T.insert(2)} \\
\includegraphics[width=1.0\textwidth]{./trees/img1.png}
{\tt T.delete(49)} \\
\includegraphics[width=1.0\textwidth]{./trees/img2.png}
{\tt T.delete(35)}\\
\includegraphics[width=1.0\textwidth]{./trees/img3.png}
{\tt T.insert(85)}\\
\includegraphics[width=1.0\textwidth]{./trees/img4.png}
{\tt T.delete(84)}\\
\includegraphics[width=1.0\textwidth]{./trees/img5.png}
\problempart % Problem 1c
\begin{itemize}
  \item rotating right at node 16 - \textbf{Not height balanced} \\
\includegraphics[width=1.0\textwidth]{./trees/img5.png}
  \item rotating left at node 16 - \textbf{Not height balanced} \\
\includegraphics[width=1.0\textwidth]{./trees/img6.png} 
  \item rotating left at node 37 - \textbf{height balanced} \\
\includegraphics[width=1.0\textwidth]{./trees/img7.png}
  \item rotating right at 37 is not possible.
\end{itemize}
\end{problemparts}

\newpage
\problem  % Problem 2

\begin{problemparts}
\problempart % Problem 2a
\textbf{min-heap} \\
\includegraphics[width=1.0\textwidth]{./heaps/img1.png}
\problempart % Problem 2b
\textbf{max-heap} \\
\includegraphics[width=1.0\textwidth]{./heaps/img2.png}
\problempart % Problem 2c
  \textbf{neither min-heap nor max-heap} \\
\includegraphics[width=1.0\textwidth]{./heaps/img3.png}
\includegraphics[width=1.0\textwidth]{./heaps/img4.png}
\includegraphics[width=1.0\textwidth]{./heaps/img5.png}
\includegraphics[width=1.0\textwidth]{./heaps/img6.png}
\problempart % Problem 2d
\textbf{min-heap} \\
\includegraphics[width=1.0\textwidth]{./heaps/img7.png}
\end{problemparts}

\newpage
\problem  % Problem 3

\begin{problemparts}
\problempart % Problem 3a
  create a max-heap keyed by the score $s_i$ of each node in $O(|A|)$ and start
  deleting the maximum element $s_i$ in $O(\log{|A|})$ for k times, so it takes $O(k\log{|A|})$. 
  this algorithm takes $O(|A| + k\log{|A|})$ time.
\problempart % Problem 3b
we should traverse the $n_x$ nodes and return their registration numberes using max-heap property,
so it takes $O(n_x)$ time.
\begin{itemize}
  \item if whe current node is a greater than $x$ return its registration number, and check 
    its children as a recursive call.
  \item if the current node is not greater than $x$ you should stop because all children of it is 
    less than $x$ by max-heap property.
\end{itemize}
\end{problemparts}

\newpage
\problem  % Problem 4
we can achieve this using the following data structures: 
\begin{itemize}
  \item max-heap stores all solar farms $(s_i, c_i)$ keyed by their capacity $c_i$.
    
  \item create a hash table $B$ stores all buildings $(b_i, d_i)$ hashed by their address $d_i$, each building 
    saves the solar farm $s_i$ in the max-heap it is connected to.
    
  \item create a hash table $SA$ contains all solar farms hashed by their address $s_i$, which points to 
    another hash table $SAC$ which contains all buildings connected to the solar farm hashed by its address $b_i$.
    
\end{itemize}

\begin{itemize}
\item {\tt initailiz(S)}: create the max-heap (priority queue), the hash table, and the hash table of 
  the buildings, and the hash table of the solar farms, in $O(n)_a$ time worst case.

\item {\tt power\_on($b_i, d_j$)}: pick the building from the hash table in $O(1)_a$ time, and connect it 
  to the solar farm with maximum capacity by subtract $d_i$ from $c_i$ then maintain the max-heap property
    in $O(log{n})$ time worst case, finally assign this solar farm to this building in $B$ 
    in $o(1)_e$ and add the building to its solar farm in $SA$ in the hash table $SAC$ in $o(1)_e$ time.

\item {\tt power\_off($b_i$)}: simply we can reverse the process of {\tt power\_on($b_i, d_j$)}, 
    find the building in $B$ and add its $d_i$ to the connected solar farm in the max-heap and 
    and maintain the max-heap property in $O(log{n})$ time worst case, remove this building from the hash table $SA$ 
    in $O(1)_e$ time, and finally remove the building from the hash table $SAC$ in $O(1)_e$ time.

\item {\tt customers($s_i$)}: find the solar farm in $SA$ in $O(1)_e$ time, and return all buildings connected
    to this solar farm in $O(k)$ time worst case.
    
\end{itemize}
  

\newpage
\problem  % Problem 5


\newpage
\problem  % Problem 6
\begin{problemparts}
\problempart % Problem 6a
lazy to
\problempart % Problem 6b
write the solution 
\problempart % Problem 6c
I've just implemented it.

\problempart
solution: 
\begin{lstlisting}
from Set_AVL_Tree import BST_Node, Set_AVL_Tree

class Key_Val_Item:
    def __init__(self, key, val):
        self.key = key
        self.val = val

    def __str__(self): 
        return "%s,%s" % (self.key, self.val)

class Part_B_Node(BST_Node):
    def subtree_update(A):
        super().subtree_update()
        A.sum = A.item.val
        if A.left:
            A.sum += A.left.sum
        if A.right:
            A.sum += A.right.sum
        
        middle= A.item.val
        if A.left:
            left = A.left.max_prefix
            middle += A.left.sum
        else:
            left = -10000000000
        if A.right:
            right = middle + A.right.max_prefix
        else:
            right = -1000000000

        A.max_prefix = max(left, right, middle)
        if left == A.max_prefix:
            A.max_prefix_key= A.left.max_prefix_key
        elif right == A.max_prefix:
            A.max_prefix_key = A.right.max_prefix_key
        else:
            A.max_prefix_key = A.item.key

class Part_B_Tree(Set_AVL_Tree):
    def __init__(self): 
        super().__init__(Part_B_Node)

    def max_prefix(self):
        '''
        Output: (k, s) | a key k stored in tree whose
                       | prefix sum s is maximum
        '''
        k, s = 0, 0
        if self.root:
            k, s = self.root.max_prefix_key, self.root.max_prefix
        return (k, s)

def tastiest_slice(toppings):
    '''
    Input:  toppings | List of integer tuples (x,y,t) representing 
                     | a topping at (x,y) with tastiness t
    Output: tastiest | Tuple (X,Y,T) representing a tastiest slice 
                     | at (X,Y) with tastiness T
    '''
    B = Part_B_Tree()   # use data structure from part (b)
    X, Y, T = 0, 0, 0
    toppings.sort(key=lambda x: x[1])
    for x, y, t in toppings:
        B.insert(Key_Val_Item(x, t))
        if B.max_prefix()[1] > T:
            X, Y, T =  B.max_prefix()[0], y, B.max_prefix()[1]
    return (X, Y, T)
  \end{lstlisting}
\end{problemparts}

\end{problems}

\end{document}
