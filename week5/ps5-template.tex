%
% 6.006 problem set 5 solutions template
%
\documentclass[12pt,twoside]{article}

\input{macros-sp20}
\newcommand{\theproblemsetnum}{5}

\title{6.006 Problem Set \theproblemsetnum}

\begin{document}

\handout{Problem Set \theproblemsetnum}

\setlength{\parindent}{0pt}
\medskip\hrulefill\medskip

{\bf Name:} Your Name

\medskip

{\bf Collaborators:} Name1, Name2

\medskip\hrulefill

%%%%%%%%%%%%%%%%%%%%%%%%%%%%%%%%%%%%%%%%%%%%%%%%%%%%%
% See below for common and useful latex constructs. %
%%%%%%%%%%%%%%%%%%%%%%%%%%%%%%%%%%%%%%%%%%%%%%%%%%%%%

% Some useful commands:
%$f(x) = \Theta(x)$
%$T(x, y) \leq \log(x) + 2^y + \binom{2n}{n}$
% {\tt code\_function}


% You can create unnumbered lists as follows:
%\begin{itemize}
%    \item First item in a list
%        \begin{itemize}
%            \item First item in a list
%                \begin{itemize}
%                    \item First item in a list
%                    \item Second item in a list
%                \end{itemize}
%            \item Second item in a list
%        \end{itemize}
%    \item Second item in a list
%\end{itemize}

% You can create numbered lists as follows:
%\begin{enumerate}
%    \item First item in a list
%    \item Second item in a list
%    \item Third item in a list
%\end{enumerate}

% You can write aligned equations as follows:
%\begin{align}
%    \begin{split}
%        (x+y)^3 &= (x+y)^2(x+y) \\
%                &= (x^2+2xy+y^2)(x+y) \\
%                &= (x^3+2x^2y+xy^2) + (x^2y+2xy^2+y^3) \\
%                &= x^3+3x^2y+3xy^2+y^3
%    \end{split}
%\end{align}

% You can create grids/matrices as follows:
%\begin{align}
%    A =
%    \begin{bmatrix}
%        A_{11} & A_{21} \\
%        A_{21} & A_{22}
%    \end{bmatrix}
%\end{align}

% You can include images and PDFs as follows:
% \includegraphics[width=0.5\textwidth]{img.jpg}

\begin{problems}

\problem  % Problem 1

\begin{problemparts}
\problempart % Problem 1a
\includegraphics[width=0.5\textwidth]{./imgs/img1.png}
\problempart % Problem 1b
\hfill
\begin{lstlisting}
  Adj ={
    "A" : ["B"], 
    "B" : ["C", "D"],
    "C" : ["E", "F"],
    "D" : ["E", "F"],
    "E" : [],
    "F" : ["E", "D"],
  }
\end{lstlisting}
\newpage
\problempart % Problem 1c
\textbf{BFS : [A, B, C, D, E, F]} \\
\includegraphics[width=0.5\textwidth]{./imgs/img2.png} \\
\textbf{DFS : [A, B, C, E, F, D]} \\
\includegraphics[width=0.5\textwidth]{./imgs/img3.png} 
\problempart % Problem 1d
  we can remove \textbf{(F, D)} which gives a \textbf{DAG} with a topological order of \textbf{[A, B, C, D, F, E]}\\
  and \textbf{(D, F)} which gives a \textbf{DAG} with a topological order of \textbf{[A, B, C, F, D, E]}\\
\end{problemparts}

\newpage
\problem  % Problem 2
Construct a graph that contains the buildings and plants as its vertices and the edges are the wires in $O(n^4)$
we want to install the generator at the plant connected to the largest number of buildings.

this can be done by running a BFS from the plant and finding the largest number of buildings in a connected component.
  G takes $O(n^4)$ time to construct. and BFS takes $O(n^4)$ time, so the overall time is $O(n^4)$.
\newpage
\problem  % Problem 3
  Construct a graph $G = (V, E)$, $V$ is the set of friends which every two short-circuiting 
  friends are connected by a single an edge in $E$. to make this party possible we need G to be a bipartite graph.
  if $G$ is a bipartite graph, then: 
  \begin{itemize}
    \item G is 2-colorable. 
    \item G does not contain any cycles of odd length.
  \end{itemize}
  
  The following algorithm will do the job:
  \begin{itemize}
    \item run \textbf{Full-BFS} on $G$ and color evey vertex at even level with the same color of the source vertex.
    \item check every edge in $E$ to see if it connects two vertices with the same color, 
      if so then $G$ is not a bipartite graph and this party is not possible, else $G$ is a bipartite graph.
  \end{itemize}

  Time complexity: O(n)
  \begin{itemize}
  \item Constructing G is $O(n)$, because we have $n$ edges, and vertices are $O(n)$
    because it is at most $2n$ vertices.
  \item Running Full-BFS is $O(n)$, since it is linear in the graph size which is $O(n)$.
  \item Checking every edge is $O(n)$.
  \end{itemize}

\newpage
\problem  % Problem 4
  Construct a graph $G = (V, E)$, $V$ is all squares that owned by a farmer $\implies$ (not a river square). 
  and $E$ is the set of edges that connect two squares which are owned by the same farmer. \\
  Run \textbf{BFS} on $G$ going through the discribed edges starting from a source node $s$ that is connected 
  to every vertex in one of the rivers, so it returns the shortest path from $s$ to an adjacent vertex to the other river.
  
  
  

  

\newpage
\problem  % Problem 5

\newpage
\problem  % Problem 6

\begin{problemparts}
\problempart % Problem 6a
\problempart % Problem 6b
\problempart % Problem 6c
\problempart Submit your implementation to {\small\url{alg.mit.edu}}.
\end{problemparts}

\end{problems}

\end{document}
