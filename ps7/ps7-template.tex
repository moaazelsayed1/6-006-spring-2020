%
% 6.006 problem set 7 solutions template
%
\documentclass[12pt,twoside]{article}

\input{macros-sp20}
\newcommand{\theproblemsetnum}{7}

\title{6.006 Problem Set 7}

\begin{document}

\handout{Problem Set \theproblemsetnum}

\setlength{\parindent}{0pt}
\medskip\hrulefill\medskip

{\bf Name:} Your Name

\medskip

{\bf Collaborators:} Name1, Name2

\medskip\hrulefill

%%%%%%%%%%%%%%%%%%%%%%%%%%%%%%%%%%%%%%%%%%%%%%%%%%%%%
% See below for common and useful latex constructs. %
%%%%%%%%%%%%%%%%%%%%%%%%%%%%%%%%%%%%%%%%%%%%%%%%%%%%%

% Some useful commands:
%$f(x) = \Theta(x)$
%$T(x, y) \leq \log(x) + 2^y + \binom{2n}{n}$
% {\tt code\_function}


% You can create unnumbered lists as follows:
%\begin{itemize}
%    \item First item in a list
%        \begin{itemize}
%            \item First item in a list
%                \begin{itemize}
%                    \item First item in a list
%                    \item Second item in a list
%                \end{itemize}
%            \item Second item in a list
%        \end{itemize}
%    \item Second item in a list
%\end{itemize}

% You can create numbered lists as follows:
%\begin{enumerate}
%    \item First item in a list
%    \item Second item in a list
%    \item Third item in a list
%\end{enumerate}

% You can write aligned equations as follows:
%\begin{align}
%    \begin{split}
%        (x+y)^3 &= (x+y)^2(x+y) \\
%                &= (x^2+2xy+y^2)(x+y) \\
%                &= (x^3+2x^2y+xy^2) + (x^2y+2xy^2+y^3) \\
%                &= x^3+3x^2y+3xy^2+y^3
%    \end{split}
%\end{align}

% You can create grids/matrices as follows:
%\begin{align}
%    A =
%    \begin{bmatrix}
%        A_{11} & A_{21} \\
%        A_{21} & A_{22}
%    \end{bmatrix}
%\end{align}

% You can include images and PDFs as follows:
% \includegraphics[width=0.5\textwidth]{img.jpg}

\begin{problems}

\problem  % Problem 1
\\
  \textbf{Subproblem}: $x(i)$ is the maximum number of delegations from day $i$ to day $n$. (suffixes), $i \in {1, \dots , n + 1}$ \\
  \textbf{Relate}: 
  \begin{itemize}
    \item we choose whether to campaign on day $i$ or not.
    \item take the maximum from $z$ and $d$ within days $i, i+1, i+2$. 
    \item $x(i) = max{(d_i + z_{i-1)} + z_{i-2} + x(i + 3)), (z_i + x(i + 1))}$  
  \end{itemize}
  \textbf{Topological Ordering:} $i$ is decreasing, and $x(i)$ requires $x(i+1)$ and $x(i+3)$. \\
  \textbf{Base:} 
  \begin{itemize}
    \item $x(n+1) = 0$
    \item $x(n) = d_n$
    \item $x(n-1) = max{(d_{n-1} + z_n), (d_n + z_{n-1})}$
  \end{itemize}
  \textbf{Original problem:}
  \begin{itemize}
    \item check whether $x(0)$ is greater than $\lfloor D / 2 \rfloor + 1$ and determine if she win or not.
  \end{itemize}

  \textbf{Time:} 
  \begin{itemize}
    \item sum $D$ of all delegations from day $i$ to day $n$ is $O(n)$.
    \item $n$ Subproblems each takes $O(1)$ time. 
    \item total time is $O(n)$.
  \end{itemize}
\newpage
\problem  % Problem 2
\begin{itemize}
  \item sort tigers by their age in decreasing order. $T$
  \item sort cages by their capacity in increasing order. $C$
  \item using dynamic programming, match the tigers $T$ with subsequance of $C$ with minimum discomfort $s_i - c_j$.
\end{itemize}
  \textbf{Subproblem:}
  \begin{itemize}
    \item $x|(i, j)$: min total discomfort, matching $T[i:]$ with $C[j:]$.
    \item $i \in {0, 1, \dots, n}$
    \item $j \in {0, 1, \dots, n^2}$
  \end{itemize}

  \textbf{Relate:}
  \begin{itemize}
    \item $d(i,j) = s_i - c_j \geq 0$
    \item $x(i, j) = min{d(i, j) + x(i + 1, j +1), x(i, j+1)}$
  \end{itemize}
  
  \textbf{Topological Ordering:} $i$ and $j$ are decreasing, and $x(i, j)$ requires $x(i + 1, j + 1)$ and $x(i, j + 1)$. \\

  \textbf{Base:}
  \begin{itemize}
    \item $x(n, j) = 0$
    \item $x(i, n^2) = \infty$
  \end{itemize}

  \textbf{Original problem:}
  \begin{itemize}
    \item $x(0, 0)$ is the minimum total discomfort.
    \item store parent of each node in a matrix $P$ to reconstruct the solution.
  \end{itemize}

  \textbf{Time:}
  \begin{itemize}
    \item $O(n^3)$ Subproblems each takes $O(1)$ time.
    \item $O(n^3)$ time 
  \end{itemize}

\newpage
\problem  % Problem 3
\textbf{Subproblem:} \\
$x(u, t): $ contains weights of all paths from $u$ to $t$ for $u \in V$. \\
$L(u, n): $ number of odd paths from $u$ to $n$ \\
\textbf{Relate:} 
\begin{itemize}
  \item $x(u, t)$ = $\sum_{j \in Adj^+(u)} x(u + 1, t, n) + w(u, j)$
  \item $L(u, n)$ = $L(u, n) + 1$ for every odd path from $u$ to $n$ in $x(u, t)$.
\end{itemize}

\textbf{Topological Ordering:} $x(u, t)$ requires $x(u + 1, t, n)$ which is calculated before $x(u, t)$. \\

  \textbf{Base:}
  \begin{itemize}
    \item $x(t, t) = 0$
    \item $L(i, j) = 0$
  \end{itemize}

  \textbf{Original problem:}
  \begin{itemize}
    \item $L(s, t)$ is the number of odd paths from $s$ to $t$.
  \end{itemize}

  \textbf{Time:}
  \begin{itemize}
    \item $|V|$ Subproblem each takes $2\sum_{u \in v} deg^+(v) = O(|V| + |E|)$
  \end{itemize}
\newpage
\problem  % Problem 4
same as \textbf{Coins} problem in lecture 16 but choose a subarray every time rather than a single coin.

\newpage
\problem  % Problem 5

\begin{problemparts}
\problempart % Problem 5a
\problempart Submit your implementation to {\small\url{alg.mit.edu}}.
\end{problemparts}

\end{problems}

\end{document}
