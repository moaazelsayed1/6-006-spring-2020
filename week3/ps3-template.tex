%
% 6.006 problem set 3 solutions template
%
\documentclass[12pt,twoside]{article}

\input{macros-sp20}
\newcommand{\theproblemsetnum}{3}

\title{6.006 Problem Set \theproblemsetnum}

\begin{document}

\handout{Problem Set \theproblemsetnum}

\setlength{\parindent}{0pt}
\medskip\hrulefill\medskip

{\bf Name:} Your Name

\medskip

{\bf Collaborators:} Name1, Name2

\medskip\hrulefill

%%%%%%%%%%%%%%%%%%%%%%%%%%%%%%%%%%%%%%%%%%%%%%%%%%%%%
% See below for common and useful latex constructs. %
%%%%%%%%%%%%%%%%%%%%%%%%%%%%%%%%%%%%%%%%%%%%%%%%%%%%%

% Some useful commands:
%$f(x) = \Theta(x)$
%$T(x, y) \leq \log(x) + 2^y + \binom{2n}{n}$
% {\tt code\_function}


% You can create unnumbered lists as follows:
%\begin{itemize}
%    \item First item in a list
%        \begin{itemize}
%            \item First item in a list
%                \begin{itemize}
%                    \item First item in a list
%                    \item Second item in a list
%                \end{itemize}
%            \item Second item in a list
%        \end{itemize}
%    \item Second item in a list
%\end{itemize}

% You can create numbered lists as follows:
%\begin{enumerate}
%    \item First item in a list
%    \item Second item in a list
%    \item Third item in a list
%\end{enumerate}

% You can write aligned equations as follows:
%\begin{align}
%    \begin{split}
%        (x+y)^3 &= (x+y)^2(x+y) \\
%                &= (x^2+2xy+y^2)(x+y) \\
%                &= (x^3+2x^2y+xy^2) + (x^2y+2xy^2+y^3) \\
%                &= x^3+3x^2y+3xy^2+y^3
%    \end{split}
%\end{align}

% You can create grids/matrices as follows:
%\begin{align}
%    A =
%    \begin{bmatrix}
%        A_{11} & A_{21} \\
%        A_{21} & A_{22}
%    \end{bmatrix}
%\end{align}

% You can include images and PDFs as follows:
% \includegraphics[width=0.5\textwidth]{img.jpg}

\begin{problems}

\problem  % Problem 1

\begin{problemparts}
\problempart % Problem 1a
 \includegraphics[width=0.5\textwidth]{img1.png}

\problempart % Problem 1b
13 as a brute force solution.
\end{problemparts}

\newpage

\problem  % Problem 2

\begin{problemparts}
\problempart % Problem 2a
  they should choose their IDs such that $|(ak_1 + b) - (ak_2 - b)| = cn$ where c is any positive integer numbere 
  which implies that $k_1 \equiv k_2 \mod n$ ,so it's guaranted.
\problempart % Problem 2b
the privious approach will not work as we devide by $n$, but this division helps to make adjacent $k_1$ and $k_2$
collide, so it's guaranted if $k_1 - k_2 < 2$.
\problempart % Problem 2c
  the probability of collisions in this case is $1/n$, then if $n > 1$ it's not guaranted.
\end{problemparts}

\newpage

\problem  % Problem 3

\begin{problemparts}
\problempart % Problem 3a
Let’s assume that each string is stored sequentially in a contiguous chunk 
of memory, in an encoding such that the numerical representation of each charac-
ter is bounded above by some constant number k (e.g. $127$ in case of ascii code),
 where the numerical representation of one 
character $c_i$ is smaller than that of another character $c_j$ if $c_i$ comes before $c_j$ in the 
  English alphabet. Then each string can be thought of as an integer between $0$ and u = $k^{64\log_4{n}} 
  = O(n^{64\log_4{k}}) = O(n^{O(1)})$ ,stored in a constant number of machine words, so 
  can be sorted using radix sort in worst-case $O(n + n \log_n{n^{O(1)}}) = O(n)$ time.

\problempart % Problem 3b
  as the number of years is bounded by some range we can use radix sort to sort it in $O(n)$ time.
\problempart % Problem 3c
  multiply all numbers by $n^3$, then we have integers in range $[0,4n^3]$, use radix sort 
  to sort them in $O(n)$ time.
\problempart % Problem 3d
  use \textit{merge sort} to sort them in $O(nlog{n})$ as we can only compare every pair to determine which
  one is older.
\end{problemparts}

\newpage

\problem  % Problem 4

\begin{problemparts}
\problempart % Problem 4a
  if the largest number of reams in a box is known (in $O(n)$) we can use radix sort to sort them 
  (saving their original indicies in $(b_i, i)$) in $O(n)$ time then we can use \textit{two-finger algorithm} 
  as follows
  \begin{itemize}
      \item if $b_i + b_j < r$ increment the left pointer.
      \item if $b_i + b_j > r$ increment the right pointer.
      \item if $b_i + b_j = r$ check if $|i - j| < \frac{n}{10}$.
        \begin{itemize}
      \item if $|i - j| = \frac{n}{10}$ then there is a close pair and algorithm terminates.
      \item if $|i - j| \neq \frac{n}{10}$ there is no close pair, since which pointer moves 
       $b_i + b_j \neq r$ as the array is sorted and \textbf{no two boxes contain the same number of reams}
       ans algorithm terminates.
        \end{itemize}
  \end{itemize}
  if the largest number of reams is unknown we can hash them in $O(n)$ time saving their indicies like $(b_i, i)$
  and loop over the boxes and search for $(r - b_i)$ in the hash table in $O(1)$ and this takes $O(n)_{exp}$.
  \problempart % Problem 4b
the first approach in part \textbf{(a)} fits well in this case with a modification that evey box has 
  number of reams $> n$ do not consider it in the new data structure $[(b_1, 1), (b_2, 2), \cdots]$.

\end{problemparts}

\newpage

\problem  % Problem 5

\begin{problemparts}
\problempart % Problem 5a
  we aim to structure a hash table by hashing evey k-length substring in $A$ so we can search for it in $O(1)$, 
  since $A$ and $K$ is given we can calcutale the frequancy array of substring of length $K$ as follows: 
  \begin{enumerate}
    \item use use array of length 26 initialized to zero.
    \item create a frequancy array for every k-length substring in A.
    \item create a hash table by hashing the frequancy array(tuple) and insert it in form $(S, i)$ where $i$ is a counter
      to determine number of anagrams of this weight.
    \item erase the first element in $K$ and append the next letter in $A$ to $K$ and hash its sum of weights. 
    \item repeat step \textbf{(4)} till the end of $A$.
  \end{enumerate}
  given string $B$ with length $k$ we can calcutale its frequancy array in $O(K)$, search for it in the 
  k hash table and return $i$ in $O(1)_exp$.
\problempart % Problem 5b
  we can hash every frequancy array of string $S_i$ in $\textbf{\textit{S}}$ in $O(nk)$, calcutale frequancy arrays for every k-length substring in \textbf{\textit{T}}
  in and hash them $O(\textbf{\textit{T}})$ and search for it in the hash table in $O(1)_exp$.
  \newpage
\problempart 
\begin{lstlisting}
def freq(str):
    arr = [0] * 26
    for c in str:
        arr[ord(c) - ord('a')] += 1
    
    return arr


def count_anagram_substrings(T, S):
    '''
    Input:  T | String
            S | Tuple of strings S_i of equal length k < |T|
    Output: A | Tuple of integers a_i:
              | the anagram substring count of S_i in T
    '''
    A = [0] * len(S) 
    frq = [0] * 26
    hsh = {}
    k = len(S[0])
    for i in range(len(T)):
        frq[ord(T[i]) - ord('a')] += 1;
        if i > k - 1:
            frq[ord(T[i - k]) - ord('a')] -= 1;
        if i >= k - 1:
            key = tuple(frq) # tuple is a hashable type
            if key in hsh:
                hsh[key] += 1 
            else:
                hsh[key] = 1 

    
        
    for i in range(len(S)):
        arr = freq(S[i])
        key = tuple(arr)
        if key in hsh:
            A[i] = hsh[key]


                
    

    return tuple(A)

\end{lstlisting}
\end{problemparts}

\end{problems}

\end{document}
