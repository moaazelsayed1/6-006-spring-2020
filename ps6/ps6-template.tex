%
% 6.006 problem set 6 solutions template
%
\documentclass[12pt,twoside]{article}

\input{macros-sp20}
\newcommand{\theproblemsetnum}{6}

\title{6.006 Problem Set \theproblemsetnum}

\begin{document}

\handout{Problem Set \theproblemsetnum}

\setlength{\parindent}{0pt}
\medskip\hrulefill\medskip

{\bf Name:} Your Name

\medskip

{\bf Collaborators:} Name1, Name2

\medskip\hrulefill

%%%%%%%%%%%%%%%%%%%%%%%%%%%%%%%%%%%%%%%%%%%%%%%%%%%%%
% See below for common and useful latex constructs. %
%%%%%%%%%%%%%%%%%%%%%%%%%%%%%%%%%%%%%%%%%%%%%%%%%%%%%

% Some useful commands:
%$f(x) = \Theta(x)$
%$T(x, y) \leq \log(x) + 2^y + \binom{2n}{n}$
% {\tt code\_function}


% You can create unnumbered lists as follows:
%\begin{itemize}
%    \item First item in a list
%        \begin{itemize}
%            \item First item in a list
%                \begin{itemize}
%                    \item First item in a list
%                    \item Second item in a list
%                \end{itemize}
%            \item Second item in a list
%        \end{itemize}
%    \item Second item in a list
%\end{itemize}

% You can create numbered lists as follows:
%\begin{enumerate}
%    \item First item in a list
%    \item Second item in a list
%    \item Third item in a list
%\end{enumerate}

% You can write aligned equations as follows:
%\begin{align}
%    \begin{split}
%        (x+y)^3 &= (x+y)^2(x+y) \\
%                &= (x^2+2xy+y^2)(x+y) \\
%                &= (x^3+2x^2y+xy^2) + (x^2y+2xy^2+y^3) \\
%                &= x^3+3x^2y+3xy^2+y^3
%    \end{split}
%\end{align}

% You can create grids/matrices as follows:
%\begin{align}
%    A =
%    \begin{bmatrix}
%        A_{11} & A_{21} \\
%        A_{21} & A_{22}
%    \end{bmatrix}
%\end{align}

% You can include images and PDFs as follows:
% \includegraphics[width=0.5\textwidth]{img.jpg}

\begin{problems}

\problem  % Problem 1
|
\begin{problemparts}
\problempart % Problem 1a
  \textbf{DAG Relaxation}: [(a, b), (a, d), (d, b), (d, e), (b, e), (b, c), (b, f), (e, f), (f, c)] \\
  \\
  \textbf{Dijkstra}: [(a,b), (a, d), (d, b), (b, c), (b, e), (b, f), (f, c), (e, f), (c, e)] \\
\problempart % Problem 1b
  % creat table of with 2 rows and 7 columns
  \begin{tabular}{c|c c c c c c}
    v & a & b & c & d & e & f \\
  \hline
    $\delta(s, v)$ & 0 & 2 & 3 & 6 & 5 & 4 \\
  \end{tabular}
\end{problemparts}

\newpage
\problem  % Problem 2
General shortest path problem in which we can use Bellman-ford, knowing that there is always some path between every 
two vertices with at most $k$ edges, implies that $|V|$ is upper bounded by $|E|$ $\rightarrow$ $|v| = O(|E|)$, and 
we can perform \textbf{DAG Relaxation} for $k$ levels. 
\begin{itemize}
  \item \textbf{If G has no negative cycles:} 
    Let $v \in V$ be any vertex. Consider path $p = \textless v_0, v_1, . . . , v_k⟩$ from $v_0 = s$ to $v_k = v$
    that is a shortest path with minimum number of edges. No negative weight cycles $\rightarrow$ p is simple $\rightarrow$
    $k \leq |V| - 1$. \\
    After 1 pass through $E$, we have $d[v_1] = \delta(s, v_1)$, because we will relax the edge $(v_0, v_1)$ in this pass
    and we can’t find a shorter path than this shortest path. 
    After 2 passes through $E$, we have $d[v_2] = \delta(s, v_2)$, 
     because in the second pass we will relax the edge $(v_1, v_2)$.
    After $i$ passes through $E$, we have $d[v_i] = \delta(s, v_i)$.
    After $k \leq |V| - 1$ passes through $E$, we have $d[v_k] = d[v] = \delta(s, v)$. 
    this takes $O(k|E|)$. 
  \item \textbf{If G has negative cycles:} we can perform the previous algorithm and then do \textbf{Bellman-ford} check
    After $k$ passes, if we find an edge that can be relaxed, it means that the current
    shortest path from $s$ to some vertex is not simple and vertices are repeated. Since this
    cyclic path has less weight than any simple path the cycle has to be a negative-weight
    cycle.
\end{itemize}
  \textbf{Bellman-ford} takes $O(|V|(|V| + |E|))$ as in this case we just relax on $k$ levels and $|v| = O(|E|)$ 
  so $O(K|E|)$.
\newpage
\problem  % Problem 3
  construct graph $G_1 = (V, E)$, $V$ is the clearings $c_i$ with the oroginal elevations $e_i$ and $E$ the slopes, 
  same way construct graph $G_2 = (V, E)$, $V$ is the clearings $c_i$ with the elevations $e_i$ after all dynamite 
  detonations and $E$ the slopes, compine $G_1$ and $G_2$ in graph $G$ with directed edges with zero-weight length from $D_1$ towards $D_2$. \\
  \textbf{Claim:} G is a \textbf{DAG}. \\
  \textbf{proof:} as Bames will only traverse slopes so as to decrease her elevation, implies that all edges go from 
  a higher $c_i$ to lower $c_j$ $\Rightarrow$ directed, this also guarantees that Bames will not need to revisit a vertex 
  more than once as he goes downhill only. \\

  Run \textbf{DAG Relaxation} from $L_1$ to some vertex $T$ connected to $s_1$ and $s_2$ with 
  edge with zero-weight length in $O(|E| + |v|) = O(n)$.
\newpage
\problem  % Problem 4

\newpage
\problem  % Problem 5

\newpage
\problem  % Problem 6

\end{problems}

\end{document}
