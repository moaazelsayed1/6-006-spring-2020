%
% 6.006 problem set 1 solutions template
%
\documentclass[12pt,twoside]{article}

\input{macros-sp20}
\newcommand{\theproblemsetnum}{1}

\title{6.006 Problem Set 1}

\begin{document}

\handout{Problem Set \theproblemsetnum}

\setlength{\parindent}{0pt}
\medskip\hrulefill\medskip

{\bf Name:} Moaaz Elsayed

\medskip

% {\bf Collaborators:} Name1, Name2

\medskip\hrulefill

%%%%%%%%%%%%%%%%%%%%%%%%%%%%%%%%%%%%%%%%%%%%%%%%%%%%%
% See below for common and useful latex constructs. %
%%%%%%%%%%%%%%%%%%%%%%%%%%%%%%%%%%%%%%%%%%%%%%%%%%%%%

% Some useful commands:
%$f(x) = \Theta(x)$
%$T(x, y) \leq \log(x) + 2^y + \binom{2n}{n}$
% {\tt code\_function}


% You can create unnumbered lists as follows:
%\begin{itemize}
%    \item First item in a list
%        \begin{itemize}
%            \item First item in a list
%                \begin{itemize}
%                    \item First item in a list
%                    \item Second item in a list
%                \end{itemize}
%            \item Second item in a list
%        \end{itemize}
%    \item Second item in a list
%\end{itemize}

% You can create numbered lists as follows:
%\begin{enumerate}
%    \item First item in a list
%    \item Second item in a list
%    \item Third item in a list
%\end{enumerate}

% You can write aligned equations as follows:
%\begin{align}
%    \begin{split}
%        (x+y)^3 &= (x+y)^2(x+y) \\
%                &= (x^2+2xy+y^2)(x+y) \\
%                &= (x^3+2x^2y+xy^2) + (x^2y+2xy^2+y^3) \\
%                &= x^3+3x^2y+3xy^2+y^3
%    \end{split}
%\end{align}

% You can create grids/matrices as follows:
%\begin{align}
%    A =
%    \begin{bmatrix}
%        A_{11} & A_{21} \\
%        A_{21} & A_{22}
%    \end{bmatrix}
%\end{align}

% You can include images and PDFs as follows:
% \includegraphics[width=0.5\textwidth]{img.jpg}

\begin{problems}

\problem  % Problem 1

\begin{problemparts}
\problempart % Problem 1a
  \begin{flalign*}
    f_1 &= \log{n^n} = O(n) &\\
    f_2 &= (\log{n})^n  = O((\log{n})^n) &\\
    f_3 &= \log{n^6006} = O(\log{n}) &\\
    f_4 &= (\log{n})^6006 = O((\log{n})^c) &\\
    f_5 &= \log{\log{6006n}} = O(\log{\log{n}}) &
  \end{flalign*}
  solution:
  \begin{align*}
        (f_5, f_3, f_4, f_1, f_2) \\
  \end{align*}
\problempart % Problem 1b
solution: 
  \begin{align*}
        (f_1, f_2, f_5, f_4, f_3) \\
  \end{align*}

\problempart % Problem 1c
solution: 
  \begin{align*}
    (\{f_2, f_5\}, f_4, f_1, f_3) \\
  \end{align*}

\problempart % Problem 1d
solution: 
  \begin{align*}
    (f_5, f_2, f_1, f_3, f_4) \\
  \end{align*}

\end{problemparts}

\newpage
\problem  % Problem 2

\begin{problemparts}
\problempart % Problem 2a
   iterate over the given data structure starting from index {\tt i} to {\tt(i + k - 1) // 2}, swap {\tt i} with (i + k - 1) 
   using two variables {\tt x1} and {\tt x2} to store the returned values from {\tt delete\_at(i)} and {\tt delete\_at(i + k - 1)} 
   then use {\tt insert\_at(i, x2)} and {\tt insert\_at(i + k - 1, x1)} to complete the swap.
   This recursive procedure makes no more than k/2 = O(k) recursive calls before reaching a base case,
   doing O(log n) work per call, so the algorithm runs in O(k log n) time. 
 \begin{lstlisting}
def reverse(D, i, k):
    if k < 2:
        return
    
    x1 = D.delete_at(i);
    x2 = D.delete_at(i + k - 1);
    D.insert_at(i, x2);
    D.insert_at(i + k - 1, x1);
    reverse(D, i + 1, k - 2)

\end{lstlisting}


\problempart % Problem 2b
  we can solve it recusively by setting the base case that is k == 0 then do nothing and recusively use 
  {\tt delete\_at(i + k - 1)} ans store the returned value in {\tt x1} then we check if {\tt j} if bigger 
  than {\tt i} we do {\tt insert\_at(j - 1, x1)} as deleting an element make a shift by one in the desired 
  index to insert at else if not {\tt j > i} we {\tt insert\_at(j, x1)}, then call {\tt move(D, i, k - 1, j)}
  This recursive procedure makes no more than k = O(k) recursive calls before reaching a base case,
  doing O(log n) work per call, so the algorithm runs in O(k log n) time. 
 
 \begin{lstlisting}
def move(D, i, k, j):
    if(k == 0):
        return

    x1 = D.delete_at(i + k - 1);

    if j > i:
        D.insert_at(j - 1, x1)
    else:
        D.insert_at(j, x1)


    move(D, i, k - 1, j)
\end{lstlisting}
\end{problemparts}

\newpage
\problem  % Problem 3

\problem  % Problem 4

\begin{problemparts}
\problempart % Problem 4a
  To support {\tt insert\_first(x)} in $O(1)$ time we can use {\tt L.head} pointer by make {\tt x.next} equals the 
  head pointer then make {\tt L.head} points to the node {\tt x}. 
  This algorithm does oconstant-time operations in so it runs in O(1). \\ 
  \\
  To support {\tt insert\_last(x)} in $O(1)$ time we can use {\tt L.tail} pointer by make {\tt L.tail.next} points 
  to the Node {\tt x} then make {\tt L.tail} points to the node {\tt x}.
  This algorithm does oconstant-time operations in so it runs in O(1). \\
  \\
  To support {\tt delete\_first()} in $O(1)$ time we make {\tt L.head} points to {\tt L.head.next}. 
  (assuming that {\tt L.head} exits) 
  This algorithm does oconstant-time operations in so it runs in O(1). \\
  \\
  To support {\tt delete\_last()} in $O(1)$ time we make {\tt L.tail} points to {\tt L.tail.prev}.
  (assuming that {\tt L.tail} exits)
  This algorithm does oconstant-time operations in so it runs in O(1). \\
  \\
\problempart % Problem 4b
  \begin{itemize}
    \item make the head pointer of {\tt L2} points to {\tt x1}.
    \item make the tail pointer of {\tt L2} points to {\tt x2}.
    \item if {\tt x1} is the head
      \begin{itemize}
        \item make {\tt x1.prev} equals {\tt x2.next}
      \end{itemize}
    \item if {\tt x1} is \textbf{NOT} the head
      \begin{itemize}
        \item make {\tt x1.prev.next} equals {\tt x2.next}
      \end{itemize}
    \item if {\tt x2} is the tail
      \begin{itemize}
          \item make the tail equals {\tt x1.prev}
      \end{itemize}
    \item if {\tt x1} is \textbf{NOT} the tail
      \begin{itemize}
        \item make {\tt x2.next.prev} equals {\tt x1.prev}
      \end{itemize}
    \item set {\tt x1.prev} and {\tt x2.next} to {\tt None}
    \item return {\tt L2}
 \end{itemize}
\problempart % Problem 4c
  \begin{itemize}
    \item make ${\tt L_2.head.prev}$ equals to {\tt x}. 
    \item save {\tt x.next} in {\tt xn} variable.
    \item make {\tt x.next} equals {\tt L2.head}.
    \item make {\tt L2.tail.next} equals {\tt xn}.
    \item if x is the tail of L1
      \begin{itemize}
        \item make {\tt L1.tail} equals {\tt L2.tail}.
      \end{itemize}
    \item if x is \textbf{not} the tail of L1
      \begin{itemize}
        \item make {\tt xn.prev} equals {\tt L2.tail}.
      \end{itemize}
  \end{itemize}

\problempart 
\begin{lstlisting}

class Doubly_Linked_List_Node:
    def __init__(self, x):
        self.item = x
        self.prev = None
        self.next = None

    def later_node(self, i):
        if i == 0: return self
        assert self.next
        return self.next.later_node(i - 1)

class Doubly_Linked_List_Seq:
    def __init__(self):
        self.head = None
        self.tail = None

    def __iter__(self):
        node = self.head
        while node:
            yield node.item
            node = node.next

    def __str__(self):
        return '-'.join([('(%s)' % x) for x in self])

    def build(self, X):
        for a in X:
            self.insert_last(a)

    def get_at(self, i):
        node = self.head.later_node(i)
        return node.item

    def set_at(self, i, x):
        node = self.head.later_node(i)
        node.item = x

    def insert_first(self, x):
        node = Doubly_Linked_List_Node(x)
        if self.head is None:
            self.head = node
            self.tail = node
        else:
            node.next = self.head
            self.head.prev = node
            self.head = node
            


    def insert_last(self, x):
        node = Doubly_Linked_List_Node(x)
        if self.head is None:
            self.head = node
            self.tail = node
        else:
            self.tail.next = node
            node.prev = self.tail
            self.tail = node

            

    def delete_first(self):
        x = self.head.item
        self.head = self.head.next
        # if we the first ans last elements are the same node
        if self.head is None:
            self.tail = None
        else:
            self.head.prev = None

        return x

    def delete_last(self):
        x = self.tail.item
        self.tail = self.tail.prev 
        # if we the first ans last elements are the same node
        if self.tail is None:
            self.head = None
        else:
            self.tail.next = None

        return x

    def remove(self, x1, x2):
        L2 = Doubly_Linked_List_Seq()
        L2.head = x1
        L2.tail = x2
        if x1 == self.head:
            x1.prev = x2.next
        else:
            x1.prev.next = x2.next

        if x2 == self.tail:
            self.tail = x1.prev
        else:
            x2.next.prev = x1.prev

        x1.prev = None
        x2.next = None
        return L2

    def splice(self, x, L2):
        L2.head.prev = x
        xn = x.next
        x.next = L2.head
        L2.tail.next = xn
        if x == self.tail:
            self.tail = L2.tail
        else:
            xn.prev = L2.tail
\end{lstlisting}
\end{problemparts}

\end{problems}

\end{document}
